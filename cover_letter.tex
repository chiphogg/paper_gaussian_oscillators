%%%%%%%%%%%%%%%%%%%%%%%%%%%%%%%%%%%%%%%%%
% Plain Cover Letter
% LaTeX Template
% Version 1.0 (28/5/13)
%
% This template has been downloaded from:
% http://www.LaTeXTemplates.com
%
% Original author:
% Rensselaer Polytechnic Institute 
% http://www.rpi.edu/dept/arc/training/latex/resumes/
%
% License:
% CC BY-NC-SA 3.0 (http://creativecommons.org/licenses/by-nc-sa/3.0/)
%
%%%%%%%%%%%%%%%%%%%%%%%%%%%%%%%%%%%%%%%%%

%----------------------------------------------------------------------------------------
%	PACKAGES AND OTHER DOCUMENT CONFIGURATIONS
%----------------------------------------------------------------------------------------

\documentclass[11pt]{letter} % Default font size of the document, change to 10pt to fit more text

\usepackage{newcent} % Default font is the New Century Schoolbook PostScript font 
%\usepackage{helvet} % Uncomment this (while commenting the above line) to use the Helvetica font

% Margins
\topmargin=-1in % Moves the top of the document 1 inch above the default
\textheight=8.5in % Total height of the text on the page before text goes on to the next page, this can be increased in a longer letter
\oddsidemargin=-10pt % Position of the left margin, can be negative or positive if you want more or less room
\textwidth=6.5in % Total width of the text, increase this if the left margin was decreased and vice-versa

\let\raggedleft\raggedright % Pushes the date (at the top) to the left, comment this line to have the date on the right

\begin{document}

%----------------------------------------------------------------------------------------
%	ADDRESSEE SECTION
%----------------------------------------------------------------------------------------

\begin{letter}{%
    Thomas~C.~M.~Lee \\
    Department of Statistics \\
    University of California, Davis  \\
    4118 Math Sci Bldg, One Shields Ave  \\
    Davis, CA 95616
}

%----------------------------------------------------------------------------------------
%	YOUR NAME & ADDRESS SECTION
%----------------------------------------------------------------------------------------

\begin{center}
\large\bf Dr.~Charles~R.~Hogg~III \\ % Your name
%\vspace{20pt} \hrule height 1pt % If you would like a horizontal line separating the name from the address, uncomment the line to the left of this text
Google, Inc. \\ 6425 Penn Avenue \#700 \\ Pittsburgh, PA 15206 \\ (412) 449-9111 % Your address and phone number
\\ \texttt{chogg@google.com}
\end{center} 
\vfill

\signature{Charles~R.~Hogg~III} % Your name for the signature at the bottom

%----------------------------------------------------------------------------------------
%	LETTER CONTENT SECTION
%----------------------------------------------------------------------------------------

\opening{Dear Dr.~Lee,} 

I wish to submit the attached paper, ``Visualizing uncertain curves and surfaces via Gaussian Oscillators,'' for consideration to be published in the \textit{Journal of Computational and Graphical Statistics}.
The paper contains 5 figures, 5862 words, and no tables.
Many papers in this journal deal with curves or surfaces (for instance, regression or Gaussian Processes), whose uncertainty is inherently challening to visualize.
Animations are a useful and beautiful approach which meet this challenge, and provide important complementary information which other techniques omit.
Since my paper advances these animations in several ways, I believe it will be of great interest to this journal's readership.

I emphasize the following novel and useful results.
This is the first paper to investigate the quality of \textit{individual} animations, rather than \textit{populations} of animations.
This is critically important, since real visualizations typically contain only a single animation.
It also introduces a new animation technique, whose motion is fluid and natural, and whose animation frames are all on equal footing (no special ``keyframes'' are singled out).
Finally, it paves the way for future developments, by identifying these animations as a special class of Gaussian processes (which I have called ``Gaussian oscillators'' for ease of reference).

Allow me to suggest several reviewers who I expect will find my work useful, and who will be able to critique it in detail.
Philipp Hennig (\texttt{phennig@tuebingen.mpg.de}) has used animations to teach Gaussian Processes in summer and winter school sessions since 2013.
John Skilling (\texttt{skilling@eircom.net}) developed the first statistically correct animation technique in 1991.
Charles Ehlschlaeger (\texttt{ehlschl1@illinois.edu}), Ashton Shortridge (\texttt{ashton@msu.edu}), and Michael Goodchild (\texttt{good@geog.ucsb.edu}) applied Dr. Skilling's technique to elevation data in a highly cited paper from 1997.

I confirm that this manuscript has been submitted solely to this journal and that it is not published, in press, or submitted elsewhere.
I confirm that this research complies with all relevant ethical guidelines and legal requirements.
I confirm that I have prepared the complete text of the paper, that I am the sole author, and that I have seen, read, and understood your guidelines on copyrights.
Finally, I confirm that I have no conflict of interest.

Thank you very much for your time and consideration.
I look forward to hearing back from you.

\closing{Sincerely yours,}


%\encl{Curriculum vitae, employment form} % List your enclosed documents here, comment this out to get rid of the "encl:"

%----------------------------------------------------------------------------------------

\end{letter}

\end{document}
